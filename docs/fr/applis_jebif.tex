\documentclass[10pt,a4paper]{book}

% Langues et caractères spéciaux

\usepackage[utf8x]{inputenc}
\usepackage{ucs}
\usepackage[frenchb]{babel}
\usepackage[T1]{fontenc}
\usepackage{amsmath}
\usepackage{amsfonts}
\usepackage{amssymb}

% Mise en page
\pagestyle{empty}
% Marges
\usepackage[left=3cm,right=3cm,top=3.5cm,bottom=3.5cm]{geometry}
% Alinéa
\setlength{\parindent}{1cm}
% Langues et caractères spéciaux

\usepackage[utf8x]{inputenc}
\usepackage{ucs}
\usepackage[frenchb]{babel}
\usepackage[T1]{fontenc}
\usepackage{amsmath}
\usepackage{amsfonts}
\usepackage{amssymb}

% Mise en page
\pagestyle{empty}
% Marges
\usepackage[left=3cm,right=3cm,top=3.5cm,bottom=3.5cm]{geometry}


\author{Nolwenn Lavielle}
\title{Applications JeBiF}

\begin{document}

\maketitle

\tableofcontents

\chapter*{Introduction}

\emph{JeBiF}, ou Jeunes Bioinformaticiens de France, est une association nationale dont le but est de fédérer 

\part{BioInfuse}

\chapter{User documentation}


\chapter{Documentation développeur}

\section{Scripts Python, dans bioinfuse/}

\subsection{admin.py}

Permet de gérer ce qui sera affiché dans le panneau d'administration de Django. Nécessaire pour créer les différents éléments dans la base de données.

\subsection{forms.py}

Permet de créer et gérer les formulaires qui seront utilisés dans l'interface utilisateur de l'application.

\subsection{models.py}

Permet de créer et interagir avec les différentes tables de la base de données de l'application.

\subsection{tests.py}

Vide. Pourra permettre de lancer des tests avec les outils internes de Django pour vérifier que l'application a le comportement attendu.

\subsection{parameters.py.dist}

Contient les paramètres nécessaires pour se connecter à un compte Daylimotion.

\subsection{upload\_movie.py}

Contient les différents essais réalisés pour comprendre comment interagir avec l'API de Daylimotion pour envoyer une vidéo et récupérer des informations.

\subsection{urls.py}

Contient les liens vers les différentes pages de l'application.

\subsection{views.py}

Contient les fonctions qui afficheront les données souhaitées dans les templates de l'applications.


\chapter{Déploiement \& Maintenance}



%\makeindex

\end{document}
